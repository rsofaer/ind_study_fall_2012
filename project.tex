\documentclass{article}

\usepackage{graphicx}
\usepackage{amssymb}
\usepackage{lastpage}
\usepackage{epstopdf}
\usepackage{fancyhdr}
\DeclareGraphicsRule{.tif}{png}{.png}{`convert #1 `dirname #1`/`basename #1 .tif`.png}

\newcommand{\cAuthor}{Raphael Sofaer}
\newcommand{\cTitle}{Implementing Low-Stretch Spanning Trees}
\pagestyle{fancy}
\lhead{\cAuthor}                                                 %
\rhead{\cTitle}  %
\lfoot{\lastxmark}                                                      %
\cfoot{}                                                                %
\rfoot{Page\ \thepage\ of\ \pageref{LastPage}}                          %
\renewcommand\headrulewidth{0.4pt}                                      %
\renewcommand\footrulewidth{0.4pt}        

\title{\cTitle}
\author{\cAuthor}
\begin{document}
\maketitle

\section*{Abstract}
We implement an algorithm found in "Lower Stretch Spanning Trees", by Spielman et al., and evaluate its stretch performance on a variety of graphs.
\section*{Introduction}
In the last twenty years, there has been a growing effort to simplify graphs with trees - usually spanning trees - of low 'stretch', in order to simplify corresponding laplacian matricies and thereby rapidly solve linear problems involving symmetric and diagonally dominant matricies.
In 1995, Alon, Karp, Peleg and West\cite{AKPG} built a zero sum game in order to investigate the k-server problem\cite{k-server}, and the payoff in their game was generalized by [3] into 

\section*{Conclusion}
\end{document}
