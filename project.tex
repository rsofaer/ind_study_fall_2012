\documentclass{article}

\usepackage{graphicx}
\usepackage{amssymb}
\usepackage{lastpage}
\usepackage{epstopdf}
\usepackage{fancyhdr}
\DeclareGraphicsRule{.tif}{png}{.png}{`convert #1 `dirname #1`/`basename #1 .tif`.png}

\newcommand{\cAuthor}{Raphael Sofaer}
\newcommand{\cTitle}{Implementing Low-Stretch Spanning Trees}
\pagestyle{fancy}
\lhead{\cAuthor}                                                 %
\rhead{\cTitle}  %
\lfoot{\lastxmark}                                                      %
\cfoot{}                                                                %
\rfoot{Page\ \thepage\ of\ \pageref{LastPage}}                          %
\renewcommand\headrulewidth{0.4pt}                                      %
\renewcommand\footrulewidth{0.4pt}        

\title{\cTitle}
\author{\cAuthor}
\begin{document}
\maketitle

\section*{Abstract}
We implement an algorithm found in "Lower Stretch Spanning Trees", by Spielman et al., and evaluate its stretch performance on a variety of graphs.
\section*{Introduction}
\subsection{Approximating with Trees}
In the last twenty years, there has been a growing effort to simplify graphs with trees - usually spanning trees - of low 'stretch', in order to simplify corresponding laplacian matricies and thereby rapidly solve linear problems involving symmetric and diagonally dominant matricies.
In 1995, Alon, Karp, Peleg and West\cite{AKPG} built a zero sum game in order to investigate the k-server problem\cite{k-server}, and the payoff in their game was adapted by Spielman et al.\cite{lower-stretch} into a more pure measure of how closely a spanning subtree approximates a graph.

\begin{table}
    \begin{tabular}{|l|l|l|l|}
    \hline
    Author                      & Year & Stretch Metric                                                                              & Average Stretch Upper Bound                             \\ \hline
    Alon, Karp, Peleg, West     & 1995 & cost(e) = {0 if e \in T, the weight of the unique cycle formed by adding e to T otherwise.} &                         exp(O(sqrt(log(n)*log(log(n)))) \\ \hline
    Elkin, Emek, Spielman, Teng & 2005 & For each edge e=(u,v) in G, stretch(e) = TreeDist(u,v)/length(e)                            & O(log(n)*log(n)*log(log(n)))                            \\ \hline
    \end{tabular}
\end{table}
\subsection{Solving with Approximate Trees}
In parallel to the improving tightness of graph-driven approximations to matricies, the solvers that use these approximations have also been getting faster and in some cases, simpler.
\section*{Conclusion}
\end{document}
